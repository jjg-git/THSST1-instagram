%%%%%%%%%%%%%%%%%%%%%%%%%%%%%%%%%%%%%%%%%%%%%%%%%%%%%%%%%%%%%%%%%%%%%%%%%%%%%%%%%%%%%%%%%%%%%%%%%%%%%%
%
%   Filename    : chapter_1.tex 
%
%   Description : This file will contain your Research Description.
%                 
%%%%%%%%%%%%%%%%%%%%%%%%%%%%%%%%%%%%%%%%%%%%%%%%%%%%%%%%%%%%%%%%%%%%%%%%%%%%%%%%%%%%%%%%%%%%%%%%%%%%%%

\chapter{Introduction}
\label{sec:intro}    %--note: labels help you with hyperlink editing (using your IDE)

This chapter gives an overview of the purpose of our study and the motivations for pursuing it. We outline our goals, limitations, and the methods that we intend to use in our research.

%%This chapter builds the motivation for the conduct of the research through a description of the background of the study or the current state of technology. This is followed by the research objectives, the scope and limitations, and significance of the research. The chapter ends with the research methodology describing the different activities to be conducted to achieve the goals of this research.%%


%%
%% --- 1.1 Background of the Study --- %%
%%

\section{Background of the Study}
\label{sec:overview}

%
%   NOTE: You have to delete/replace the unnecessary paragraphs with your own text.
%


%%This section gives the reader an overview of the specific technology or field in the international or local setting. The information regarding the technology or field should be contemporary and not
%%based on outdated sources. Discussion must not be too technical or too detailed.

%%Follow the inverted pyramid to describe each of the following. Allocate one (1) paragraph per level in the pyramid. 

% Describe the research area
%Research area ...
%This study's main research area is on personality computing. Personality computing is related to artificial intelligence and personality psychology by studying one's personality through a variety of computational techniques from different sources (usually text, multimedia, and social networks). Among the three main problems personality computing addresses; automatic personality recognition, perception, and synthesis \citep[p.~280]{Vinciarelli2014}; this study focuses on automatic personality recognition.

% Describe the domain (where will you apply the topic)

Personality is defined as “patterns of thought, emotion, and behavior” \citep{Vinciarelli2014}. For purposes of personality computing, personality is broken down into various characteristics that can be measured quantitatively \citep{Vinciarelli2014}. The most widely used model in this field is the Big Five Personality Model or Big Five Inventory (BFI), which comprises the traits of extraversion, agreeableness, conscientiousness, neuroticism, and openness. Other well-known models include the Myers–Briggs Type Indicator (MBTI) and the Cattell Sixteen Personality Factor (16PF). Using these models, the traditional method of measuring personality is by using a multiple-choice questionnaire \citep{farnadi2018}, wherein each choice corresponds to a value for each personality characteristic. These values are then aggregated to obtain the final results, which may take the form of numerical scores for each trait or categorical classifications. Understanding personality plays a significant role in fields such as psychology, as this information can be used in studies of mental health and human behavior. It has also garnered increasing interest in the fields of social science, business, education, and healthcare \citep{Miller2014} as it offers valuable insights into individual differences and supports more personalized approaches in these domains.

However, these traditional methods of identifying personality are time-consuming and may be inconvenient for participants \citep{chen2016}, prompting the need for alternative methods to obtain these results. This leads us to the field of Automatic Personality Recognition. Automatic personality recognition (APR) aims to infer an individual’s personality based on their digital footprint. APR has been traced back to the work of \citet{Pennebaker1999} wherein they asked the question of \textit{can language use reflect personality style?} APR studies typically use inputs such as text, audio, or visual data, along with an individual’s personality trait scores, to train machine learning models. These models aim to analyze patterns and identify correlations in order to form accurate personality predictions, which may be either categorical or continuous (regression-based) in nature.

% Give a synthesis of previous work

% Synthesis of previous studies...
%\citet{Pennebaker1999} and \citet{Mairesse2007} had demonstrated the different ways of detecting personalities and emotions by analyzing the text with various language models. Due to that, social media had became a common place for the research of personality detection. The PagkataoKo dataset has been made with social media activity on Instagram and X (formerly known as Twitter). There were more studies using X's data than Instagram due to X being a more prominent platform with its text-based posts. However, \citet{ferwerda2016}, \cite{Reece2017-qw} and \cite{Harris2019-gq} made studies that showed analyzing photos was possible. Thus, researching on Instagram became more meaningful in this manner.

(WIP) In the field of APR, researchers have explored various machine learning techniques—particularly deep learning models such as deep neural networks (DNNs) and support vector machines (SVMs)—to link digital content to these personality traits. While early studies predominantly analyzed text-based features, more recent work has expanded to include multimodal data, such as images and videos, reflecting the evolving nature of online content. Previous studies have found success in the implementation of deep learning and deep neural networks in the field of APR \citep{Zhao2022}. The standard approach involves feature extraction, followed by machine learning techniques to associate extracted features with personality traits. 

With the widespread adoption of social networking sites (SNS), personal content such as images, captions, and interactions has become publicly available on a large scale, providing accessible behavioral data for personality analysis and greater opportunity for research. Prior research suggests that user-generated content on these platforms often reflects personal expression in spontaneous and unintended ways \citep{Vinciarelli2014}, making them valuable for computational personality inference. Thus, many studies have taken to conducting personality analysis using data from personal social media profiles. Text was a common modality in their research, though a smaller number of papers have also explored image-based analysis. 

Despite growing research in APR, most studies have predominantly focused on Western populations. Although some progress has been made in other languages, research specific to Filipino APR remains relatively limited. This area presents unique challenges, such as frequent code-switching among users and the presence of diverse regional languages \citep{tighe_acorda_2022}. Existing APR in this field have primarily focused on either the usage of social media itself or text-based linguistic analysis of social media posts made by Filipinos. There is a heavier focus on textual data as a modality, particularly X (formerly Twitter) data, while image-based and multimodal approaches remain largely underexplored. There are a number of studies that show improved performance of models when using multimodal approaches rather than unimodal \citep{Christian2021}, yet this approach remains largely underexplored in the context of Filipino automatic personality recognition (APR). Most existing studies focus solely on single-modal data, particularly text data  \citep{Mehta2020}. However, platforms like Instagram offer a rich source of multimodal data, specifically images and text, that could notably improve personality analysis. Exploring this untapped resource appears promising, especially considering the previous success of multimodal approaches in non-Filipino contexts.

% State the problem and the research question

% Paragraph containing the Problem Statement or the Research Question.

This study seeks to explore image-based APR methods tailored for Filipino users by leveraging Instagram data, integrating psychological analysis and computational techniques. The study aims to address the question: \textit{How effective is multimodal Instagram data as a source for recognizing manifestations of Filipino personality traits?}

% This section ends with a discussion on the problem/s faced by or that still exist in the specific technology or field (e.g., limitations of existing software or algorithms). It should not contain your research objectives or goal; instead, the problem statement would lead to the research objectives to be stated in 1.2.

% \subsection{Figures}

% Often times, a graph, illustration, screenshot, or any image can help a reader better understand what we say in text. In academic writing, we call them \textbf{figures}. Please add as many figures as necessary to help build your narrative, but do not go overboard. 

% You can add figures in JPG or PNG format as shown in Figure \ref{fig:disneystock}. All figures should also have a descriptive caption. As a general principle, your caption should adequately describe what is shown in the figure, and not just short texts. If we remove all the surrounding text, the reader should still understand what your figure is. 

% All figures should be referred to at least once in the surrounding paragraphs. Make sure that you explain what the figure is all about, and that you refer to your figure. You can also use the surround paragraphs to highlight key insights or parts of your figure. For example, \figref{fig:disneystock} shows a graph of the performance of Disney stock from the 1980s to 2012.

% %--- the following example shows how to include a figure in PNG format
% \begin{figure}[t]                %-- use [t] to place figure at top, [b] to place at the bottom, [h] for here
%    \centering                    %-- use this to center the figure
%    \includegraphics{DisneyChart.png}      %-- include image file named as "disneychart.png" 
%    \caption{Disney's stock price chart from the 1980s up to 2012. The top chart shows the stock price in each month. The bottom chart shows the change in volume of transactions. The bars are colored based on the type of movement that happened in a month.}
%     \label{fig:disneystock}
% \end{figure}

% \subsection{References}

% Some notes on citing references. When using APA format, the author-date method of citation is followed. This means that the author's last name and the year of publication for the source should appear in the text, and a complete reference should appear in the reference list.

% %
% % Examples:
% %     	Smith (1970) compared reaction times . . .
% %     	In a recent study of reaction times (Smith, 1970), . . .   
% %     	In 1970, Smith compared reaction times . . .
% %	Smith, et al., (1970) compared reaction times . . .
% %     	In a recent study of reaction times (Smith, et al., 1970), . . .  
% %     	In 1970, Smith, et al., compared reaction times . . .
% %

% Here are some examples on how to do the referencing (note author's name and years are different from commented examples). For APA citation details, refer
% to \url{http://www.ctan.org/tex-archive/biblio/bibtex/contrib/apacite/}. 

% \begin{itemize}
%  \item \citeA{kartch:2000:ERA} compared reaction times...
%  \item In a recent study of reaction times \cite{kartch:2000:ERA}...
%  \item In \citeyearNP{kartch:2000:ERA}, \citeauthor{kartch:2000:ERA} compared reaction times...
%  \item \shortciteA{fedkiw:2001:VSO} compared reaction times... 
%  \item In a recent study of reaction times \cite{fedkiw:2001:VSO}...
%  \item In \citeyearNP{fedkiw:2001:VSO}, \shortciteauthor{fedkiw:2001:VSO}, compared reaction times...
% \end{itemize}

% The following are references from journal articles \cite{Park:2006:DSI, Pellacini:2005:LAH, sako:2001:SSB}.  Here's an MS thesis document \cite{yee:2000:SSA}, and this is from a PhD dissertation \cite{kartch:2000:ERA}. For a book, reference is given as \cite{parke:1996:CFA}.  Proceedings from a conference samples are \cite{Jobs95, fedkiw:2001:VSO,levoy:2000:TDM}.  

% The sample bibliography file named \textbf{myreferences.bib} is from the
% SIGGRAPH \LaTeX template.  You can use a text editor to view the contents of the bib file. It is your task to create your own bibliography file.  For those who downloaded papers from ACM or IEEE sites, there is a BibTeX link that you can click; thereafter, you just simply need to copy and paste the BibTeX entry into your own bibliography file.

% \subsection{Code snippet}

% The following shows how to include a program source code (or algorithm). The verbatim environment, as the name suggests, outputs text (including white spaces) as is...

% \begin{verbatim}
%                #include <stdio.h>
%                main()
%                {
%                     printf("Hello world!\n");
%                }
% \end{verbatim}


%%
%% --- 1.2 Research Objectives --- %%
%%

\section{Research Objectives}
\label{sec: objectives}
This study aims to explore the effectivity of a multimodal approach using Instagram data in the study of Filipino Automatic Personality Recognition. To achieve this, the study will pursue the following specific objectives:

\begin{enumerate}
	\item To identify data characteristics of the Instagram subset of the PagkataoKo dataset that are relevant to APR.

	\item To perform image and text feature extraction on the dataset based on previously identified characteristics.

	\item To build personality prediction models using different feature sets and machine learning algorithms.

	\item To evaluate the performance of the models using various metrics and compare the results to existing unimodal approaches.
\end{enumerate}

% This subsection states the over--all goal that must be achieved to answer the problem. Address the following: Given your research challenge or opportunity, how do you intend to solve it? What is the main outcome of your research? What kind of contribution do you want to achieve?

% The \textbf{general objective} is broken down into three or more specific objectives. The \textbf{specific objectives} are relatively smaller objectives that help you attain your general objective. You can formulate your specific objectives based on your running questions about the project. For example, the group does not have a good picture yet of the different conversation patterns between humans that can be mimicked by conversational agents. A specific objective can be "to identify different human-human conversation patterns." These objectives must be specific, measurable, attainable, realistic, and time-bounded.

% Reviewing related literature, studying a particular programming language or development tool (e.g., to study Windows/Object-Oriented/Graphics/C++ programming), and documentation to accomplish the general objective is inherent in all thesis projects and, therefore, must not be included here.


\begin{comment}
%
% IPR acknowledgement: the following sentences and examples are from Ethel Ong's slides 
%     on Research Objectives
%

How to formulate your research objectives:
1. Identify what research steps do you need to perform to achieve your general objective.
2. Identify the questions that must be answered for you to achieve your general objective.
    Thereafter, convert these questions into action statements


Example #1:

Question:
    What strategies do human educators employ in collaborative storytelling with children?

Specific Objective:
   To review existing strategies employed by language educators when sharing storytelling with children


Example #2:

Question:
   How will you represent commonsense knowledge for use by computer systems?

Specific Objective:
   To identify knowledge representation approaches used by existing story generation systems

Example #3:
Question:
   What types of storytelling knowledge are needed to generate stories?

Objective:
    To identify the different types of storytelling knowledge used in generating stories

Example #4:
Question:
    What machine learning approaches will you utilize?

Specific Objective:
    To determine existing machine learning algorithms [that can be used in training the computer system to detect cyberbullying cases] 

Example #5: 
Question:
    How will your research output be evaluated?

Specific Objective:
    To define evaluation metrics for validating the accuracy of the translation
\end{comment}


%
%  The following is the format for presenting your specific objectives; replace them with your own 
%

% The specific objectives of this research are as follows:
% \begin{enumerate}
%    \item To identify knowledge representation approaches used by existing story generation systems;
%    \item To identify the different types of storytelling knowledge used in generating stories;
%    \item To build a neural-based model for generating events comprising a story; and
%    \item To define evaluation metrics for evaluating the performance of the event generation model.
% \end{enumerate}


%%
%% --- 1.3 Scope and Limitations --- %%
%%

\section{Scope and Limitations of the Research}
\label{sec:scopelimitations}


	This study aims to develop a multimodal personality recognition framework that predicts Filipino Instagram users' Big Five personality traits using both image and text data. The following outlines and discusses the scope, limitations, and rationale for each specific research objective
	
	\subsection{Perform exploratory data analysis (EDA) on the dataset}
	
	\textbf{Why should this objective be done?} \\
	Exploratory Data Analysis (EDA) is necessary to understand the dataset's structure, distribution, and quality. It helps identify patterns, inconsistencies, and potential biases within Instagram images and captions, which will inform and improve subsequent stages of feature extraction and modeling.
	
	\textbf{Scope, limitations, and rationale:} \\
	The EDA is limited to Instagram posts that contain both images and captions, ensuring that the dataset is suitable for the intended multimodal analysis. Posts lacking either modality are excluded, as they do not contribute to the core objective of integrating text and image data. Additionally, social interaction metrics such as likes, comments, or follower counts fall outside the scope, as this study focuses solely on content-based personality cues rather than social engagement behaviors.
	
	\subsection{Structure and extract features from the image and text data present in the finalized subset}
	
	\textbf{Why should this objective be done?} \\
	Raw social media data is inherently unstructured and cannot be directly used for personality trait prediction. Effective preprocessing and feature extraction are required to convert images and captions into structured, machine-readable formats that capture relevant personality-related features.
	
	\textbf{Scope, limitations, and rationale:} \\
	For text processing, only captions written in English or Filipino will be considered, with linguistic and sentiment-based features extracted. Hashtags, emojis, user bios, and other forms of textual metadata are excluded to reduce noise and maintain focus on personality-relevant linguistic signals. For image processing, the study is limited to edge detection techniques, color composition analysis, and basic content features. Advanced image processing methods, such as deep semantic segmentation or complex object detection, are excluded to ensure methodological clarity and computational efficiency.
	
	\subsection{Train machine learning models for predicting users' Big Five personality traits}
	
	\textbf{Why should this objective be done?} \\
	Training machine learning models is essential to operationalize the proposed multimodal framework and empirically evaluate its ability to predict personality traits based on the extracted image and text features.
	
	\textbf{Scope, limitations, and rationale:} \\
	The modeling process is confined to classification-based approaches, as the goal is to assign users to discrete personality trait categories rather than predict continuous scores. Only Support Vector Machines (SVM) and XGBoost are considered, given their proven performance and compatibility with moderately sized datasets. Deep learning models, while potentially powerful, are excluded to prioritize interpretability and avoid overfitting, considering the dataset's size constraints.
	
	\subsection{Evaluate the predictive performance of the proposed multimodal framework}
	
	\textbf{Why should this objective be done?} \\
	Evaluating the framework is critical to determining whether the integration of image and text data results in improved personality prediction performance compared to unimodal models.
	
	\textbf{Scope, limitations, and rationale:} \\
	The evaluation is limited to within-dataset testing, using standard metrics such as Macro-F1 score and AUC-ROC to assess classification performance. The framework is designed specifically for Filipino Instagram users, limiting the generalizability of results to other populations, cultures, or social media platforms.
	

% This section discusses the boundaries, with respect to the objectives, of the research and the constraints within which the research will be developed. Describe what is and is not included in the scope of your research, supported by your main research question and findings of previous studies. Do not use weak excuses such as the lack of time and/or knowledge to perform the research.

% A good rule of thumb is to allocate one paragraph for each of your specific objectives that (1) contains a brief overview of the concept/theory and the purpose of doing the associated objective; and (2) includes a description of the scope/limitation of your study, and followed by brief purpose, rationale and/or justification for your decisions.

% The following should also be indicated in your Scope and Limitations (in the appropriate paragraphs matching the objectives, or as a stand-alone paragraph):
% \begin{itemize}
%    \item The profile and demographics of your target participants
%    \item Your data sources (i.e., new data, data from previous studies, data to be provided by some experts, data to be retrieved from social networks)
%    \item The specific technology platform to be utilized
%    \item The methods for collecting the data
%    \item The coverage areas or locations
%    \item The duration or time period (e.g., news articles for the year 2016-2017)
% \end{itemize}


%%
%% --- 1.4 Significance of the Research --- %%
%%

\section{Significance of the Study}
\label{sec: Significance}

This study aims to address the current research gap in Philippine Automatic Personality Recognition (APR) by exploring how multimodal data specifically, image and text content from Instagram can be used to infer personality traits among Filipino users. While most existing APR studies for Filipinos rely heavily on text-based data from platforms such as Twitter, this research expands the scope by incorporating visual analysis, introducing an image-based component into the local APR landscape.

By conducting exploratory data analysis (EDA), this study contributes to understanding the structural and demographic characteristics of Filipino Instagram data that are relevant for personality recognition. This helps establish a foundation for the appropriate development of computational models in this domain.

The study also demonstrates the practical application of feature extraction and fusion techniques for processing multimodal Instagram data in the Philippine context. Through the development and evaluation of machine learning models using combined image and text features, this research provides empirical evidence on the effectiveness of multimodal approaches compared to traditional unimodal methods.

While the primary contribution is academic, the study's findings offer broader implications for future work in culturally sensitive APR systems and personality computing. Moreover, this research provides a foundation for building more inclusive, localized models that account for the unique linguistic, cultural, and behavioral patterns of Filipino social media users.

%%This section explains why research must be done in this area.  It rationalizes the objective of the research with that of the stated problem. Avoid including sentences such as ``This research will be beneficial to the proponent/department/college'' as this is already an inherent requirement of all BS and MS thesis projects.  Focus on the research's contribution to the Computer Science field.

%%The following are guide questions that may help your formulate the significance of your research. 


%
% IPR acknowledgement: the following list of items are from Ethel Ong's slides on Significance of the Research
%
%%\begin{itemize}
%%\item  What is the relevance and contribution of your work to the computer science community? 

\section{Ethical Considerations}
\label{sec:Ethics}

The methods involved in the creation of the PagkataoKo dataset were reviewed and approved by the Research Ethics Office of De La Salle University \citep{Tighe_Acorda_Agno_Gano_Go_Santiago_Sedillo_2022}. This approval ensures that the study adheres to established ethical standards for research involving human participants, particularly in the collection and handling of social media data.

\begin{itemize}
	\item \textbf{Data Compliance:}
	\begin{itemize}
		\item All volunteers provided informed electronic consent before their data was included in the dataset. This consent process outlined the scope, purpose, and limitations of the research.
		\item Data collection complied with the official developer policies and terms of service of both Twitter and Instagram, ensuring responsible and lawful data access through their respective APIs.
	\end{itemize}
	
	\item \textbf{Privacy Protection:}
	\begin{itemize}
		\item Personally identifiable information (PII), including \texttt{UserID} and account metadata, was fully anonymized to prevent the tracing of data back to individuals.
		\item All raw data, such as original posts and media files, were stored in encrypted form and restricted to authorized personnel only. This minimizes the risk of data breaches or unauthorized access.
	\end{itemize}
	
	\item \textbf{Data Restrictions:}
	\begin{itemize}
		\item The dataset is used strictly for academic research purposes and is not intended for any commercial exploitation.
		\item Re-identification of users is explicitly prohibited, and no attempts were made to reconstruct individual profiles or trace behavior back to specific persons.
	\end{itemize}
\end{itemize}

%%\begin{itemize} 
%%\item How does your technical contributions or empirical findings advance the field or grow our body of knowledge? 
%%\item If you built a prototype of an interaction technique, interface, library, tool, or system, what is value does it add compared to existing solutions? 
%%\end{itemize}

%%\item What will be your contributions to society in general? 
%%    \begin{itemize}
%%      \item How will your main stakeholders benefit from your technical contributions or empirical findings? 
%%      \item What are the positive social or economic impacts? 
%%%%   \end{itemize}
%%\end{itemize}

\begin{comment}
If applicable, describe possible commercialization and/or innovation in your research.
\end{comment}
