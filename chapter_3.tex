%%%%%%%%%%%%%%%%%%%%%%%%%%%%%%%%%%%%%%%%%%%%%%%%%%%%%%%%%%%%%%%%%%%%%%%%%%%%%%%%%%%%%%%%%%%%%%%%%%%%%%
%
%   Filename    : chapter_3.tex 
%
%   Description : This file will contain your Theoretical Framework.
%                 
%%%%%%%%%%%%%%%%%%%%%%%%%%%%%%%%%%%%%%%%%%%%%%%%%%%%%%%%%%%%%%%%%%%%%%%%%%%%%%%%%%%%%%%%%%%%%%%%%%%%%%

\chapter{Theoretical Framework}
\label{sec:theoframework}

This research is anchored on several foundational theories and models that inform the development of a multimodal approach to automatic personality recognition, particularly within the context of Filipino Instagram users. The convergence of personality psychology, multimodal machine learning, and social media behavior forms the theoretical backbone of this study.

\section{The Big Five Personality Traits (Five-Factor Model)}
The Five-Factor Model (FFM), also known as the Big Five, is the most widely accepted framework for understanding human personality. It categorizes personality into five dimensions: Openness to Experience, Conscientiousness, Extraversion, Agreeableness, and Neuroticism (McCrae \& Costa, 1999). This model provides a structured approach to labeling and quantifying personality traits, which is essential for supervised learning in automatic personality recognition. The Big Five is used as the ground truth for evaluating personality from user-generated data.

\section{Multimodal Machine Learning Theory}
Multimodal machine learning refers to models that process and integrate multiple types of data (e.g., text, images, audio) to perform tasks more effectively than single-modality systems (Baltrušaitis et al., 2019). Instagram, as a social media platform, is inherently multimodal—users post textual captions, visual content, and other metadata. This study leverages multimodal theory to combine linguistic (e.g., captions, hashtags), visual (e.g., images, filters), and behavioral cues (e.g., posting frequency) to improve the accuracy of personality recognition.

\section{Social Media Self-Presentation Theory}
According to Goffman’s (1959) theory of self-presentation and its applications to digital platforms, individuals curate their online profiles to project a desired image. Instagram posts often reflect personality traits, whether consciously or unconsciously. This research assumes that the way Filipino users present themselves on Instagram through their language, image content, and interaction patterns offers valid signals of their underlying personality traits.

\section{Cultural Dimensions Theory}
Hofstede’s Cultural Dimensions Theory (Hofstede, 1980) suggests that culture affects behavior and communication. Since this research focuses on Filipino users, it is important to consider how Filipino cultural values (such as collectivism, high-context communication, and pakikisama) may influence both content creation and personality expression online. This cultural lens helps interpret the features extracted from Instagram data in a way that aligns with localized behavioral norms.

\section{Computational Personality Recognition (CPR)}
CPR is an interdisciplinary field combining psychology, natural language processing (NLP), computer vision, and data science to infer personality traits from digital footprints (Youyou et al., 2015). This study builds on previous CPR works by applying a multimodal strategy tailored to Filipino Instagram users, whose social media use patterns and cultural context differ from Western-centric datasets often used in CPR studies.