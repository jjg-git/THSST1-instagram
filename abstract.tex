%%%%%%%%%%%%%%%%%%%%%%%%%%%%%%%%%%%%%%%%%%%%%%%%%%%%%%%%%%%%%%%%%%%%%%%%%%%%%%%%%%%%%%%%%%%%%%%%%%%%%%
%
%   Filename    : abstract.tex 
%
%   Description : This file will contain your abstract.
%                 
%%%%%%%%%%%%%%%%%%%%%%%%%%%%%%%%%%%%%%%%%%%%%%%%%%%%%%%%%%%%%%%%%%%%%%%%%%%%%%%%%%%%%%%%%%%%%%%%%%%%%%

\begin{abstract}
%% \begin{mdframed} [style=highlight]
Filipino Automatic Personality Recognition (APR) has predominantly focused on using Twitter from the PagkataoKo dataset using only text features from tweets. While Filipino Instagram users represent a rich source of behavioral data, no current APR research leverages both image and text modalities for this demographic, despite evidence that visual and linguistic cues jointly enhance trait prediction. To address this, we propose the first multimodal APR framework for Filipino Instagram users, combining VGG-19-derived image features with bilingual (English/Tagalog) text analysis to advance beyond unimodal approaches. Our classification methodology employs SVM and XGBoost models with intermediate attention-based fusion of image-text-metadata features, systematically comparing unimodal and multimodal performance across Big Five traits while adhering to ethical guidelines established in the PagkataoKo Dataset.  
%% \end{mdframed}




%
%  Do not put citations or quotes in the abract.
%
\end{abstract}
