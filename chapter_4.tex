\chapter{Methodology}
\label{sec:methodology}

\section{Research Methodology}

Our multimodal personality recognition pipeline employs intermediate fusion to integrate image, text, and metadata features from Filipino Instagram users. The methodology consists of six key phases:

\subsection{Preprocessing}
\label{subsec:preprocessing}

\begin{itemize}
	\item \textbf{Data Aggregation:}
	\begin{itemize}
		\item Posts are grouped by \texttt{UserID} to form user-level profiles.
		\item Metadata such as \texttt{PostCount}, \texttt{Following Count}, and temporal posting patterns are extracted to serve as behavioral signals.
	\end{itemize}
	
	\item \textbf{Missing Data Handling:}
	\begin{itemize}
		\item Posts without captions are retained for image-only analysis.
		\item Missing text features are imputed using zero vectors.
		\item Metadata-derived proxies such as \texttt{PostCount} act as activity indicators.
	\end{itemize}
\end{itemize}

\subsection{Exploratory Data Analysis}
\label{subsec:eda}

\begin{itemize}
	\item \textbf{Demographic Analysis:}
	\begin{itemize}
		\item Assesses geographic distribution (e.g., Luzon, Visayas, Mindanao).
		\item Reviews age and gender representation.
	\end{itemize}
	
	\item \textbf{Visual Pattern Discovery:}
	\begin{itemize}
		\item Identifies color dominance through HSV histograms.
		\item Uses OpenCV’s Haar cascades for face detection (e.g., face count in selfies).
	\end{itemize}
	
	\item \textbf{Temporal Trends:}
	\begin{itemize}
		\item Examines correlations between posting times and personality traits (e.g., late-night activity as potential Neuroticism signal).
	\end{itemize}
\end{itemize}

\subsection{Feature Extraction}
\label{subsec:features}

\begin{itemize}
	\item \textbf{Image Features:}
	\begin{itemize}
		\item \textbf{VGG-19}: Extracts deep style and texture features from images (e.g., visual complexity).
		\item \textbf{Object Detection}: Utilizes JSON metadata to infer social context (e.g., group photos linked to Extraversion).
		\item Color features are derived using k-means clustering in HSV color space.
	\end{itemize}
	
	\item \textbf{Text Features:}
	\begin{itemize}
		\item Captions are processed using LIWC categories, extended to include Tagalog.
		\item Bilingual sentiment analysis is applied for English and Tagalog texts.
	\end{itemize}
	
	\item \textbf{Metadata Features:}
	\begin{itemize}
		\item Behavioral signals such as \texttt{Following Count}, post frequency, and time-of-day activity are captured.
	\end{itemize}
\end{itemize}

\subsection{Intermediate Feature Fusion}
\label{subsec:fusion}

To integrate the diverse feature types, we use an intermediate fusion strategy at the feature level:

\begin{enumerate}
	\item \textbf{Modality-Specific Encoding:}
	\begin{itemize}
		\item Image features: VGG-19 embeddings (4,096 dimensions).
		\item Text features: TF-IDF-weighted LIWC vectors.
		\item Metadata features: Min-max scaled numeric values.
	\end{itemize}
	
	\item \textbf{Concatenation:} A unified multimodal feature vector is formed for each user.
	
	\item \textbf{Dynamic Weighting:} An attention-based mechanism dynamically prioritizes modalities depending on trait:
	\begin{itemize}
		\item Image features are weighted more for Openness.
		\item Text features are emphasized for Neuroticism.
	\end{itemize}
\end{enumerate}

\subsection{Model Training}
\label{subsec:models}

Each personality trait is modeled independently using both single-modality and fused feature vectors.

\subsubsection{Support Vector Machines (SVM)}
\label{subsubsec:svm}

\begin{itemize}
	\item \textbf{Rationale:}
	\begin{itemize}
		\item Well-suited for high-dimensional data \cite{farnadi2018}.
		\item RBF kernel captures non-linear patterns.
		\item Performs reliably with medium-scale datasets and is robust against overfitting.
	\end{itemize}
	
	\item \textbf{Implementation:}
	\begin{itemize}
		\item Hyperparameters (\texttt{C}, \texttt{gamma}) are optimized via grid search.
		\item Class weights are balanced to address trait distribution imbalances.
	\end{itemize}
\end{itemize}

\subsubsection{XGBoost}
\label{subsubsec:xgboost}

\begin{itemize}
	\item \textbf{Rationale:}
	\begin{itemize}
		\item Handles heterogeneous and structured data efficiently \cite{chen2016}.
		\item Provides built-in tools for feature importance analysis.
	\end{itemize}
	
	\item \textbf{Implementation:}
	\begin{itemize}
		\item Early stopping is applied to prevent overfitting.
		\item Key parameters: \texttt{max\_depth=6}, \texttt{learning\_rate=0.1}, and 500 estimators.
	\end{itemize}
\end{itemize}

\subsection{Model Analysis}
\label{subsec:analysis}

We evaluate both single- and multimodal models using the following criteria:

\begin{itemize}
	\item \textbf{Evaluation Metrics:}
	\begin{itemize}
		\item Macro-F1 score to account for class imbalance.
		\item AUC-ROC to measure discriminatory power across traits.
		\item SHAP values to enhance model interpretability.
	\end{itemize}
	
	\item \textbf{Ablation Studies:}
	\begin{itemize}
		\item Assess the contribution of each modality.
		\item Compare performance of single versus fused modalities.
	\end{itemize}
\end{itemize}
